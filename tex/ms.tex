% !TeX root = ./ms.tex
\documentclass[modern]{aastex62}

% Load the corTeX style definitions
% !TeX root = ./ms.tex
% All the packages
\usepackage{url}
\usepackage{amsmath}
\usepackage{mathtools}
\usepackage{amssymb}
\usepackage{natbib}
\usepackage{graphicx}
\usepackage{calc}
\usepackage{etoolbox}
\usepackage{xspace}
\usepackage[T1]{fontenc} % https://tex.stackexchange.com/a/166791
\usepackage{textcomp}
\usepackage{ifxetex}
\ifxetex
\usepackage{fontspec}
\defaultfontfeatures{Extension = .otf}
\fi
\usepackage{fontawesome}
\usepackage{listings}
\usepackage{nicefrac}
\usepackage[bb=boondox]{mathalfa}
\usepackage{booktabs}
\usepackage{longtable}

% Shorthand for this paper
\newcommand{\starry}{\textsf{starry}\xspace}
\newcommand{\Python}{\textsf{Python}\xspace}
\newcommand{\cpp}{\textsf{C}++\xspace}
\newcommand{\bvec}[1]{{\ensuremath{\mathbf{#1}}}}
\newcommand{\xxx}[1]{{\color{red}#1}}
\DeclarePairedDelimiter\floor{\lfloor}{\rfloor}
\DeclarePairedDelimiter\ceil{\lceil}{\rceil}
\newcommand{\imag}{{\ensuremath{\mathbb{i}}}}
\newcommand{\quadquad}{\quad\quad\quad\quad}

\newcommand{\R}{\bvec{R}}
\newcommand{\AOne}{\bvec{A_1}}
\newcommand{\alm}{\bvec{a}}
\newcommand{\x}{\bvec{x}}
\newcommand{\D}{D}
\newcommand{\Doppler}{\bvec{D}}
\newcommand{\Surf}{\mathcal{S}}
\newcommand{\Curve}{\mathcal{C}}
\newcommand{\Dargs}{\bvec{d}}
\newcommand{\lmax}{\ensuremath{l_\mathrm{max}}}
\newcommand{\spot}{\texttt{SPOT}\xspace}
\newcommand{\vogtstar}{\texttt{VOGTSTAR}\xspace}
\newcommand{\kT}{\boldsymbol{\kappa}^\top}
\newcommand{\rhoT}{\boldsymbol{\rho}^\top}
\newcommand{\ylmbasis}{\boldsymbol{\psi}^\top}
\newcommand{\pbasis}{\boldsymbol{\phi}^\top}
\newcommand{\pbasisn}{\ensuremath{\phi_n}}
\newcommand{\azero}{\ensuremath{\bvec{a_0}}}

% References to text content
\newcommand{\documentname}{\textsl{article}}
\newcommand{\figureref}[1]{\ref{fig:#1}}
\newcommand{\Figure}[1]{Figure~\figureref{#1}}
\newcommand{\figurelabel}[1]{\label{fig:#1}}
\renewcommand{\eqref}[1]{\ref{eq:#1}}
\newcommand{\Eq}[1]{Equation~(\eqref{#1})}
\newcommand{\eq}[1]{\Eq{#1}}
\newcommand{\eqalt}[1]{Equation~\eqref{#1}}

% Add code, proof, and animation hyperlinks
\definecolor{linkcolor}{rgb}{0.1216,0.4667,0.7059}
\newcommand{\codeicon}{{\color{linkcolor}\faFileCodeO}}
\newcommand{\prooficon}{{\color{linkcolor}\faPencilSquareO}}
\newcommand{\codelink}[1]{\href{https://github.com/rodluger/starrynight/blob/0b2200131a5df828a979f33af6dedcc1f1009a59/tex/figures/#1.py}{\codeicon}\,\,}
\newcommand{\animlink}[1]{\href{https://github.com/rodluger/starrynight/blob/0b2200131a5df828a979f33af6dedcc1f1009a59/tex/figures/#1.gif}{\animicon}\,\,}
\newcommand{\prooflink}[1]{\href{https://github.com/rodluger/starrynight/blob/0b2200131a5df828a979f33af6dedcc1f1009a59/tex/proofs/#1.ipynb}{\raisebox{-0.1em}{\prooficon}}}
\newcommand{\cilink}[1]{\href{https://dev.azure.com/rodluger/starrynight/_build}{#1}}


% Define a proof environment for open source equation proofs
\newtagform{eqtag}[]{(}{)}
\newcommand{\currentlabel}{None}
\newenvironment{proof}[1]{%
\ifstrempty{#1}{%
\renewtagform{eqtag}[]{\raisebox{-0.1em}{{\color{red}\faPencilSquareO}}\,(}{)}%
}{%
\renewtagform{eqtag}[]{\prooflink{#1}\,(}{)}%
}%
\usetagform{eqtag}%
\renewcommand{\currentlabel}{#1}
\align%
}{%
\endalign%
\renewtagform{eqtag}[]{(}{)}%
\usetagform{eqtag}%
\message{<<<\currentlabel: \theequation>>>}%
}

% Display the runtime on Azure
\usepackage[skins]{tcolorbox}
\newtcbox{\figtimebox}{enhanced,nobeforeafter,tcbox raise=-0.8mm,boxrule=0.6pt,
  top=0.5mm,bottom=0mm,right=0mm,left=6mm,arc=1pt,boxsep=2pt,
  before upper={\vphantom{dlg}},colframe=linkcolor,coltext=linkcolor,
  fontupper=\sffamily\bfseries\tiny,colback=white,overlay={\begin{tcbclipinterior}
  \fill[linkcolor] (frame.south west)
  rectangle node[text=white,font=\sffamily\bfseries\tiny,rotate=0]{CPU} 
  ([xshift=6mm]frame.north west);\end{tcbclipinterior}}}
\robustify{\figtimebox}
\pdfstringdefDisableCommands{%
  \def\figtimebox#1{'#1'}%
}
\newcommand{\figtime}[1]{\IfFileExists{figures/#1.py.time}%
{%
\cilink{\figtimebox{\input{figures/#1.py.time}\unskip s}}
}{}}

% Define the `oscaption` command for open source figure captions
\newcommand{\oscaption}[2]{\caption{#2 \codelink{#1} \figtime{#1}}}

% Code examples
\definecolor{codegreen}{rgb}{0,0.6,0}
\definecolor{codegray}{rgb}{0.5,0.5,0.5}
\definecolor{codepurple}{rgb}{0.58,0,0.82}
\definecolor{backcolour}{rgb}{0.95,0.95,0.95}
\lstdefinestyle{mystyle}{
    backgroundcolor=\color{backcolour},
    commentstyle=\color{codegreen},
    keywordstyle=\color{magenta},
    numberstyle=\tiny\color{codegray},
    stringstyle=\color{codepurple},
    basicstyle=\small\ttfamily,
    breakatwhitespace=false,
    breaklines=true,
    captionpos=b,
    keepspaces=true,
    numbers=left,
    numbersep=5pt,
    showspaces=false,
    showstringspaces=false,
    showtabs=false,
    tabsize=2,
    aboveskip=1em,
    belowskip=1em,
    keywords=[2]{map},
    keywordstyle=[2]{\color{black!80!black}},
    upquote=true
}
\lstset{style=mystyle}

% Typography obsessions
\setlength{\parindent}{3.0ex}
\renewcommand\quad{\hskip\fontdimen3\font}

% https://tex.stackexchange.com/a/184474
\usepackage{stackengine,scalerel}
\def\lnlam{\ThisStyle{\ensurestackMath{\stackon[-2.4\LMpt]{%
  \SavedStyle\lambda}{\kern-.5pt\kern\LMpt\rule{1\LMex}{.25pt+.15\LMpt}}}}}

% Shorthand for this paper
\usepackage{xifthen}
\newcommand{\dd}{\ensuremath{\mathrm{d}}}
\newcommand{\STARRYQUADPOINTS}{100\xspace}
\newcommand{\bkappa}{\boldsymbol{\kappa}}
\newcommand{\blambda}{\boldsymbol{\lambda}}
\newcommand{\bxi}{\boldsymbol{\xi}}
\newcommand{\vmax}{{v_\mathrm{max}}}
\newcommand{\kapint}[1]{%
\sum_{i = 0}^{\frac{N - 1}{2}}
\int\limits_{\frac{1}{2}\kappa_{2i}}^{\frac{1}{2}\kappa_{2i+1}}
#1
\dd\varphi
}
\newcommand{\lamint}[1]{%
\sum_{i = 0}^{\frac{N - 1}{2}}
\int\limits_{\frac{1}{2}\lambda_{2i}}^{\frac{1}{2}\lambda_{2i+1}}
#1
\dd\varphi
}
\newcommand{\coshalfkap}[1][]{%
\ifthenelse{%
    \equal{#1}{}
}{
    \cos\left(\frac{\bkappa}{2}\right)
}{
    \cos^{#1}\left(\frac{\bkappa}{2}\right)
}}
\newcommand{\sinhalfkap}[1][]{%
\ifthenelse{%
    \equal{#1}{}
}{
    \sin\left(\frac{\bkappa}{2}\right)
}{
    \sin^{#1}\left(\frac{\bkappa}{2}\right)
}}
\newcommand{\DE}{\Delta \boldsymbol{E}(k^2, \bkappa)}
\newcommand{\DF}{\Delta \boldsymbol{F}(k^2, \bkappa)}
\newcommand{\sgn}{{\mathrm{sgn}}}
\newcommand{\xhat}{\ensuremath{\mathbf{\hat{x}}}\xspace}
\newcommand{\yhat}{\ensuremath{\mathbf{\hat{y}}}\xspace}
\newcommand{\zhat}{\ensuremath{\mathbf{\hat{z}}}\xspace}


% Bibliography stuff
\bibliographystyle{aasjournal}

% Begin!
\begin{document}

% Title
\title{Occultation Light Curves in Reflected Light}

% Author list
\author[0000-0002-0296-3826]{Rodrigo Luger}
\email{rluger@flatironinstitute.org}
\affil{Center~for~Computational~Astrophysics, Flatiron~Institute, New~York, NY}
%

\begin{abstract}
    Abstract here.
    %
    \href{https://github.com/rodluger/starrynight}{\color{linkcolor}\faGithub}
\end{abstract}

%
\section{Introduction}
\label{sec:intro}
%
This is an extension of \citet{Luger2019}.

\pagebreak

%
\section{The Problem}
\label{sec:the-problem}

As in \citet{Luger2019}, we compute the instantaneous visible flux $f$ from
a spherical body whose surface intensity is described by the spherical harmonic
coefficient vector $\mathbf{y}$ as
%
\begin{align}
    \label{eq:sTARRy}
    f = \bvec{s}^\top \bvec{A} \bvec{R}' \bvec{R} \bvec{y}
    \quad,
\end{align}
%
where $\bvec{R}$ and $\bvec{R}'$ are spherical harmonic rotation matrices,
$\bvec{A}$ is a change-of-basis matrix to a basis $\tilde{\bvec{g}}$,
and $\bvec{s}^\top$ is the \emph{solution vector}, the integral of each
of the basis functions in $\tilde{\bvec{g}}$ over
the projected visible disk of the body (which may be partially occulted by
another spherical body).
For more information on these terms, the reader is referred to
\citet{Luger2019}.

\begin{figure}[t!]
    \begin{centering}
        \includegraphics[width=0.35\linewidth]{figures/illum.pdf}
        \oscaption{illum}{%
            Illumination geometry.
            \label{fig:illum}
        }
    \end{centering}
\end{figure}

The derivations in \citet{Luger2019} assume that the coefficient vector
$\mathbf{y}$ describes the \emph{emissivity} of the body, which (in the
absence of limb darkening) is assumed to be Lambertian, i.e., all points on the
surface emit equally in all directions.
Here, we derive the solution to the occultation problem for Lambertian
reflectance, in which case the vector $\mathbf{y}$ describes the
surface \emph{reflectivity} (or \emph{Bond albedo}). Specifically, we assume
the body is illuminated by a point-like source and the radiance at any point
is proportional to the cosine of the angle between the incident light
and the surface normal. Points for which this angle is greater than
$\frac{\pi}{2}$ are unilluminated and therefore have radiance of zero;
see Figure~\ref{fig:illum}.

% case 1: 0
% case 2: sTr
% case 3: sTe * I + ~sTr
% case 4: sTe * I
% case 5: sTr - sT0 * I
% case 6: (sTe + sT0) * I + ~sTr
% case 7: (sTe - sT0) * I
% case 8: sT0 * I

\begin{figure}[t!]
    \begin{centering}
        \includegraphics[width=\linewidth]{figures/cases.pdf}
        \oscaption{cases}{%
            The eight families of cases of occultations in reflected light.
            In these figures, the night side of the occulted body is colored
            black (dark grey if occulted), and the dayside is colored blue
            (light grey if occulted). The cases in the top row involve
            configurations in which the limb of the occultor does not intersect
            with the terminator at any point, so the visible flux may be
            computed in terms of classical \starry integrals. The remaining
            cases require integration along the red boundary (the curves
            $\mathcal{P}$, $\mathcal{T}$, and $\mathcal{Q}$ of
            \S\ref{sec:integration}), which
            include the terminator. These involve the evaluation of incomplete
            elliptic integrals and are derived below.
            \label{fig:cases}
        }
    \end{centering}
\end{figure}

\section{Cases 1--4}
\label{sec:cases-1-4}
TODO.

\section{Cases 5--8}
\label{sec:cases-5-8}

\begin{figure}[t!]
    \begin{centering}
        \includegraphics[width=\linewidth]{figures/geometry.pdf}
        \oscaption{geometry}{%
            Geometry of an occultation in reflected light, corresponding
            to case 5 in Figure~\ref{fig:cases}. The surface integral over
            the blue region is computed from the antiderivatives of the
            surface intensity map counter-clockwise along the boundary curves
            $\mathcal{P}$, $\mathcal{T}$, and $\mathcal{Q}$.
            \label{fig:geometry}
        }
    \end{centering}
\end{figure}

As in \citet{Luger2019}, the elements of the solution vector $\mathbf{s}^\top$
are given by the line integral
%
\begin{align}
    \label{eq:s}
    s_{n(\mu,\nu)} = \oint \mathbf{G}_{\mu,\nu} \cdot \dd\mathbf{r}
\end{align}
%
where
%
\begin{align}
    \label{eq:G}
    \bvec{G}_{\mu,\nu} (x, y) & =
    \begin{dcases}
        %
        x^{\frac{\mu + 2}{2}}
        y^{\frac{\nu}{2}}
        \,\yhat
         & \qquad \mu, \nu \, \mathrm{even}
        \\[1em]
        %
        \frac{1-z(x, y)^3}{3(1-z(x, y)^2)}\bigg(-y \, \xhat + x \, \yhat\bigg)
         & \qquad \mu = \nu = 1
        \\[1em]
        %
        x^{l-2}
        z(x, y)^3
        \,\xhat
         & \qquad \nu \, \mathrm{odd}, \,
        \mu = 1, \,
        \frac{\mu + \nu}{2} \, \mathrm{even}
        \\[1em]
        %
        x^{l-3}
        y
        z(x, y)^3
        \,\xhat
         & \qquad \nu \, \mathrm{odd}, \,
        \mu = 1, \,
        \frac{\mu + \nu}{2} \, \mathrm{odd}
        \\[1em]
        %
        x^{\frac{\mu-3}{2}}
        y^{\frac{\nu-1}{2}}
        z(x, y)^3
        \,\yhat
         & \qquad \mathrm{otherwise,}
    \end{dcases}
\end{align}
%
and
%
\begin{align}
    \dd \bvec{r} & = -r \sin\varphi \, \dd \varphi \, \xhat +
    r \cos\varphi \, \dd \varphi \, \yhat
    \quad,
\end{align}
%
where $\varphi = \varphi(x, y)$ is the parametrized angle along the
arc of integration. Generally, there are at most three arcs bounding a given
closed surface of integration, which we will denote $\mathcal{P}$,
$\mathcal{Q}$, and $\mathcal{T}$. These are shown for a typical occultor-occulted
configuration in Figure~\ref{fig:geometry}.


\section{Performance}
\label{sec:performance}

%
\begin{figure}[h!]
    \begin{centering}
        \includegraphics[width=\linewidth]{figures/speed.pdf}
        \oscaption{speed}{%
            Evaluation time for the starry algorithm.
            \label{fig:speed}
        }
    \end{centering}
\end{figure}


\section{Integration}
\label{sec:integration}

\subsection{Definitions}
%
Define the pairwise difference operator
%
\begin{align}
    \label{eq:pairdiff}
    \Delta \boldsymbol{v} \equiv \sum_{i=0}^{\frac{N - 1}{2}}
    \left( v_{2i + 1} - v_{2i} \right)
\end{align}
%
which sums the difference of successive pairs of values in
an array $\boldsymbol{v} = \{ v_0, v_1, v_2, v_3, {\cdot\cdot\cdot}, v_N\}$.

Define the quantity
%
\begin{align}
    \label{eq:q}
    \boldsymbol{q}(k^2, \bkappa) = \sqrt{1 - \frac{\sinhalfkap[2]}{k^2}}
    \quad.
\end{align}

\subsection{$\mathcal{H}_{u,v}$}
%
The function $\mathcal{H}_{u,v}$ is given by
%
\begin{align}
    \label{eq:H}
    \mathcal{H}_{u,v}(\blambda) & =
    \lamint{
        \cos^u\varphi
        \sin^v\varphi
    }
    \quad.
\end{align}
%
The integral in this expression is the same as that in Equation (D27)
of \citet{Luger2019}, except for a change in the limits of integration.
We can compute this integral recursively given four lower boundary conditions:
%
\begin{proof}{H}
    \label{eq:Hlower}
    \mathcal{H}_{0,0}(\blambda) &= \Delta \blambda
    %
    \nonumber \\
    %
    \mathcal{H}_{1,0}(\blambda) &= \Delta \sin\blambda
    %
    \nonumber \\
    %
    \mathcal{H}_{0,1}(\blambda) &= -\Delta \cos\blambda
    %
    \nonumber \\
    %
    \mathcal{H}_{1,1}(\blambda) &= -\frac{\Delta\cos^2\blambda}{2}
    %
    \quad.
\end{proof}
%
The remaining terms may be computed by upward recursion using the
relations
%
\begin{proof}{H}
    \label{eq:Hrec1}
    \mathcal{H}_{u,v}(\blambda) &=
    \frac{
        -\Delta \left(
        \cos^{u + 1} \blambda
        \sin^{v - 1} \blambda
        \right)
        +(v - 1)\mathcal{H}_{u,v - 2}(\blambda)
    }{u + v}
\end{proof}
%
for $u < 2, v \ge 2$ and
%
\begin{proof}{H}
    \label{eq:Hrec2}
    \mathcal{H}_{u,v}(\blambda) &=
    \frac{
        \Delta \left(
        \cos^{u - 1} \blambda
        \sin^{v + 1} \blambda
        \right)
        + (u - 1)\mathcal{H}_{u - 2,v}(\blambda)
    }{u + v}
\end{proof}
%
for all remaining terms.

\subsection{$\mathcal{T}_{2}$}
%
The function $\mathcal{T}_{2}$ is given by
%
\begin{align}
    \label{eq:T2}
    \mathcal{T}_{2}(b, \bxi) & = ?
    \quad.
\end{align}

\begin{proof}{T2}
    \mathcal{T}_{2}(b, \bxi) &= -\sgn(b) \Delta \boldsymbol{f}(b, \bxi)
\end{proof}
%
where
%
\begin{proof}{T2}
    \begin{split}
        f_i(b, \xi_i) =
        \frac{1}{3}\scalebox{1.4}{\Bigg(}
        &\arctan\left( \frac{|b|\sin\xi_i}{\cos\xi_i} \right)
        %
        \\
        %
        &- \sgn\left({\sin\xi_i}\right)
        \left(
        \arctan
        \left(
            \frac{
                \left(\frac{\sin\xi_i}{1 + \cos\xi_i}\right)^2 + 2 b^2 - 1
            }{2 |b| \sqrt{1 - b^2}}
            \right)
        + |b| \sqrt{1 - b^2} \cos\xi_i
        \right)
        %
        \\
        %
        &+ \phi(b, \xi_i)
        \scalebox{1.4}{\Bigg)}
    \end{split}
\end{proof}
%
and
%
\begin{proof}{T2}
    \phi(b, \xi_i) & =
    \begin{cases}
        0                          & \qquad 0 \leq \xi_i < \frac{\pi}{2}     \\
        \pi                        & \qquad \frac{\pi}{2} \leq \xi_i < \pi   \\
        2 |b| \sqrt{1 - b^2}       & \qquad \pi < \xi_i \leq \frac{3\pi}{2}  \\
        \pi + 2 |b| \sqrt{1 - b^2} & \qquad \frac{3\pi}{2} \leq \xi_i < 2\pi
        \quad.
    \end{cases}
\end{proof}

\subsection{$\mathcal{U}_v$}
%
The function $\mathcal{U}_v$ is given by
%
\begin{align}
    \label{eq:U}
    \mathcal{U}_v(\bkappa) =
    \kapint{\cos\varphi\sin^{2v}\varphi}
    \quad.
\end{align}
%
This integral has an analytic solution for all $v$:
%
\begin{proof}{U}
    \label{eq:Usol}
    \mathcal{U}_v(\bkappa) &= \frac{\Delta \sinhalfkap[v+1]}{v + 1}
    \quad.
\end{proof}
%

\subsection{$\mathcal{I}_v$}
%
The function $\mathcal{I}_v$ is given by
%
\begin{align}
    \label{eq:I}
    \mathcal{I}_v(\bkappa) & =
    \kapint{\sinhalfkap[2v]}
    \quad.
\end{align}
%
The integral in this expression is the same as that in Equation (D38)
of \citet{Luger2019}, except for a change in the limits of integration.
As in \citet{Luger2019}, we can compute this integral recursively given
a trivial lower boundary condition:
%
\begin{proof}{I}
    \label{eq:Irec}
    \mathcal{I}_0(\bkappa) &=
    \frac{\Delta \bkappa}{2}
    %
    \nonumber \\
    %
    \mathcal{I}_v(\bkappa) &=
    \frac{1}{2v}
    \bigg(
    (2v - 1) \mathcal{I}_{v-1}(\bkappa) -
    \Delta \left(\sinhalfkap[2v - 1]\coshalfkap\right)
    \bigg)
\end{proof}
%
where the last expression is valid for all $v > 0$. We find that this algorithm
is generally stable, except when $\sin\left(\frac{\bkappa}{2}\right)$ is small. In
that limit, we evaluate $\mathcal{I}_N(\bkappa)$ by numerical integration of
Equation~(\ref{eq:I}) using Gauss-Legendre quadrature with \STARRYQUADPOINTS
points. We then recurse downward by substituting $v \rightarrow v + 1$ in
Equation~(\ref{eq:Irec}) and solving for $\mathcal{I}_v(\bkappa)$.

\subsection{$\mathcal{J}_v$}
%
The function $\mathcal{J}_v$ is given by
%
\begin{align}
    \label{eq:J}
    \mathcal{J}_v(k^2, \bkappa) =
    \kapint{
        \sin^{2v}\varphi
        \left(1 - \frac{\sin^2\varphi}{k^2}\right)^\frac{3}{2}
    }
    \quad.
\end{align}
%
The integral in this expression is again the same as that in Equation (D39)
of \citet{Luger2019}, except for a change in the limits of integration.
In that paper, we computed all terms
$\{ \mathcal{J}_0, {\cdot\cdot\cdot}, \mathcal{J}_\vmax \}$ from a three-term
recurrence relation and two boundary conditions. In the case of upward
recursion, the boundary conditions $\mathcal{J}_0$ and $\mathcal{J}_1$ were
computed analytically from the complete elliptic integrals $K(k^2)$
and $E(k^2)$. In cases where upward recursion was not numerically stable, we
evaluated $\mathcal{J}_\vmax$ and $\mathcal{J}_{\vmax-1}$
via a quickly convergent series expansion and recursed downward.

In order to solve Equation~(\ref{eq:J}), it is possible to
replace the complete elliptic integrals $K(k^2)$ and $E(k^2)$ in the lower
boundary conditions \citep[Equation D46 in ][]{Luger2019} with the
incomplete elliptic integrals $F(k^2, \kappa)$ and $E(k^2, \kappa)$,
then use the same upward
recursion relation to obtain analytic solutions for all $\mathcal{J}_v$:
%
\begin{proof}{J}
    \label{eq:Jrec}
    \mathcal{J}_0(k^2, \bkappa) &=
    \frac{1}{3} \bigg(
    2 \left(2 - \frac{1}{k^2}\right) \DE +
    \left(\frac{1}{k^2} - 1\right) \DF +
    \Delta \boldsymbol{z}_0(k^2, \bkappa)
    \bigg)
    %
    \nonumber \\
    %
    \mathcal{J}_1(k^2, \bkappa) &=
    \frac{1}{15} \bigg(
    \left(-3 k^2 + 13 - \frac{8}{k^2}\right) \DE +
    \left(3 k^2 - 7 + \frac{4}{k^2}\right) \DF +
    \Delta \boldsymbol{z}_1(k^2, \bkappa)
    \bigg)
    %
    \nonumber \\
    %
    \mathcal{J}_v(k^2, \bkappa) &=
    \frac{1}{2v + 3}
    \bigg(
    2 \left( v + 1 + (v - 1) k^2 \right) \mathcal{J}_{v - 1} -
    (2v - 3) k^2 \mathcal{J}_{v - 2}
    + \Delta \boldsymbol{z}_v(k^2, \bkappa)
    \bigg)
\end{proof}
%
%
where the last expression is valid for all $v > 1$ and
%
\begin{proof}{J}
    \label{eq:Jrec_z}
    \boldsymbol{z}_0(k^2, \boldsymbol{\kappa}) & =
    \frac{
        \sinhalfkap
        \coshalfkap
        \boldsymbol{q}(k^2, \bkappa)
    }{
        k^2
    }
    %
    \nonumber\\
    %
    \boldsymbol{z}_1(k^2, \boldsymbol{\kappa}) & =
    \left(\left(3 \sinhalfkap[2] + 4\right) - 6k^2\right)
    \boldsymbol{z}_0(k^2, \boldsymbol{\kappa})
    %
    \nonumber\\
    %
    \boldsymbol{z}_v(k^2, \boldsymbol{\kappa}) & =
    k^2
    \sinhalfkap[2v - 3]
    \coshalfkap
    \boldsymbol{q}(k^2, \boldsymbol{\kappa})^5
    \quad.
\end{proof}
%
However, in practice we find that this procedure is even more numerically
unstable than it was in \citet{Luger2019}.
To address this, we express the recurrence structure of the problem as
a tridiagonal system with one lower boundary condition $\mathcal{J}_0$
and one upper boundary condition $\mathcal{J}_\vmax$:
%
\begin{proof}{J}
    \label{eq:Jtri}
    \begin{pmatrix}
        a_0 & 1   &     &        &         &         \\
        b_1 & a_1 & 1   &        &         &         \\
            & b_2 & a_2 & 1      &         &         \\
            &     & b_0 & a_3    & 1       &         \\
            &     &     & \ddots & \ddots  & \ddots  \\
            &     &     &        & b_\vmax & a_\vmax
    \end{pmatrix}
    \begin{pmatrix}
        \mathcal{J}_1   \\
        \mathcal{J}_2   \\
        \mathcal{J}_3   \\
        \mathcal{J}_4   \\
        \cdot\cdot\cdot \\
        \mathcal{J}_{\vmax-1}
    \end{pmatrix}
    =
    \begin{pmatrix}
        c_0 - b_0 \mathcal{J}_0 \\
        c_1                     \\
        c_2                     \\
        c_3                     \\
        \cdot\cdot\cdot         \\
        c_\vmax - \mathcal{J}_\vmax
    \end{pmatrix}
\end{proof}
%
where the recursion coefficients are given by
%
\begin{proof}{J}
    \label{eq:Jtri_coeffs}
    a_v(k) &= -2\frac{(v + 1) + (v - 1) k^2}{2v + 3} \nonumber \\
    b_v(k) &= \frac{(2v - 3) k^2}{2v + 3} \nonumber \\
    c_v(k^2, \bkappa) &= \Delta
    \bigg(
    \frac{
            \boldsymbol{z}_v(k^2, \boldsymbol{\kappa})
        }{
            2v + 3
        }
    \bigg)
    \quad.
\end{proof}
%
Solving this matrix system yields values for all
intermediate $\{ \mathcal{J}_1, {\cdot\cdot\cdot}, \mathcal{J}_{\vmax - 1} \}$.
While efficient algorithms exist for solving tridiagonal problems, we obtain
far better numerical stability by instead performing traditional LU
decomposition. We find that this algorithm is stable in all the regimes that we
tested.

We evaluate the upper boundary condition $\mathcal{J}_{\vmax}$ by numerical
integration of Equation~(\ref{eq:J}) via Gauss-Legendre quadrature with
\STARRYQUADPOINTS points. While the lower boundary condition may be computed
analytically from Equation~(\ref{eq:Jrec}),
in practice we achieve better precision via numerical
integration (as above), with negligible effects on computational performance.

\subsection{$\mathcal{W}_n$}
%
The function $\mathcal{W}_n$ is given by
%
\begin{align}
    \label{eq:W}
    % TODO
    \mathcal{W}_n = \int
\end{align}
%
This integral has a closed-form solution:
%
\begin{align}
    \label{eq:Wsol}
    \mathcal{W}_n =
    \frac{\sin^{2n + 2}\left(\frac{\bkappa}{2}\right)}{2n + 5}
    \left(
    \frac{3}{n+1}
        {_2F_1}\left(-\frac{1}{2}, n + 1; n + 2; 1 - q^2\right) + 2 q^3
    \right)
\end{align}
%
where ${_2F_1}(a, b; c; z)$ is the Gauss hypergeometric function.


% by either upward recursion (stable for |1 - q^2| > 1/2) or downward
% recursion (always stable).

% Bibliography
\bibliography{bib}


\end{document}
