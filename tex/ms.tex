% !TeX root = ./ms.tex
\documentclass[modern]{aastex62}

% Load the corTeX style definitions
% !TeX root = ./ms.tex
% All the packages
\usepackage{url}
\usepackage{amsmath}
\usepackage{mathtools}
\usepackage{amssymb}
\usepackage{natbib}
\usepackage{graphicx}
\usepackage{calc}
\usepackage{etoolbox}
\usepackage{xspace}
\usepackage[T1]{fontenc} % https://tex.stackexchange.com/a/166791
\usepackage{textcomp}
\usepackage{ifxetex}
\ifxetex
\usepackage{fontspec}
\defaultfontfeatures{Extension = .otf}
\fi
\usepackage{fontawesome}
\usepackage{listings}
\usepackage{nicefrac}
\usepackage[bb=boondox]{mathalfa}
\usepackage{booktabs}
\usepackage{longtable}

% Shorthand for this paper
\newcommand{\starry}{\textsf{starry}\xspace}
\newcommand{\Python}{\textsf{Python}\xspace}
\newcommand{\cpp}{\textsf{C}++\xspace}
\newcommand{\bvec}[1]{{\ensuremath{\mathbf{#1}}}}
\newcommand{\xxx}[1]{{\color{red}#1}}
\DeclarePairedDelimiter\floor{\lfloor}{\rfloor}
\DeclarePairedDelimiter\ceil{\lceil}{\rceil}
\newcommand{\imag}{{\ensuremath{\mathbb{i}}}}
\newcommand{\quadquad}{\quad\quad\quad\quad}

\newcommand{\R}{\bvec{R}}
\newcommand{\AOne}{\bvec{A_1}}
\newcommand{\alm}{\bvec{a}}
\newcommand{\x}{\bvec{x}}
\newcommand{\D}{D}
\newcommand{\Doppler}{\bvec{D}}
\newcommand{\Surf}{\mathcal{S}}
\newcommand{\Curve}{\mathcal{C}}
\newcommand{\Dargs}{\bvec{d}}
\newcommand{\lmax}{\ensuremath{l_\mathrm{max}}}
\newcommand{\spot}{\texttt{SPOT}\xspace}
\newcommand{\vogtstar}{\texttt{VOGTSTAR}\xspace}
\newcommand{\kT}{\boldsymbol{\kappa}^\top}
\newcommand{\rhoT}{\boldsymbol{\rho}^\top}
\newcommand{\ylmbasis}{\boldsymbol{\psi}^\top}
\newcommand{\pbasis}{\boldsymbol{\phi}^\top}
\newcommand{\pbasisn}{\ensuremath{\phi_n}}
\newcommand{\azero}{\ensuremath{\bvec{a_0}}}

% References to text content
\newcommand{\documentname}{\textsl{article}}
\newcommand{\figureref}[1]{\ref{fig:#1}}
\newcommand{\Figure}[1]{Figure~\figureref{#1}}
\newcommand{\figurelabel}[1]{\label{fig:#1}}
\renewcommand{\eqref}[1]{\ref{eq:#1}}
\newcommand{\Eq}[1]{Equation~(\eqref{#1})}
\newcommand{\eq}[1]{\Eq{#1}}
\newcommand{\eqalt}[1]{Equation~\eqref{#1}}

% Add code, proof, and animation hyperlinks
\definecolor{linkcolor}{rgb}{0.1216,0.4667,0.7059}
\newcommand{\codeicon}{{\color{linkcolor}\faFileCodeO}}
\newcommand{\prooficon}{{\color{linkcolor}\faPencilSquareO}}
\newcommand{\codelink}[1]{\href{https://github.com/rodluger/starrynight/blob/0b2200131a5df828a979f33af6dedcc1f1009a59/tex/figures/#1.py}{\codeicon}\,\,}
\newcommand{\animlink}[1]{\href{https://github.com/rodluger/starrynight/blob/0b2200131a5df828a979f33af6dedcc1f1009a59/tex/figures/#1.gif}{\animicon}\,\,}
\newcommand{\prooflink}[1]{\href{https://github.com/rodluger/starrynight/blob/0b2200131a5df828a979f33af6dedcc1f1009a59/tex/proofs/#1.ipynb}{\raisebox{-0.1em}{\prooficon}}}
\newcommand{\cilink}[1]{\href{https://dev.azure.com/rodluger/starrynight/_build}{#1}}


% Define a proof environment for open source equation proofs
\newtagform{eqtag}[]{(}{)}
\newcommand{\currentlabel}{None}
\newenvironment{proof}[1]{%
\ifstrempty{#1}{%
\renewtagform{eqtag}[]{\raisebox{-0.1em}{{\color{red}\faPencilSquareO}}\,(}{)}%
}{%
\renewtagform{eqtag}[]{\prooflink{#1}\,(}{)}%
}%
\usetagform{eqtag}%
\renewcommand{\currentlabel}{#1}
\align%
}{%
\endalign%
\renewtagform{eqtag}[]{(}{)}%
\usetagform{eqtag}%
\message{<<<\currentlabel: \theequation>>>}%
}

% Display the runtime on Azure
\usepackage[skins]{tcolorbox}
\newtcbox{\figtimebox}{enhanced,nobeforeafter,tcbox raise=-0.8mm,boxrule=0.6pt,
  top=0.5mm,bottom=0mm,right=0mm,left=6mm,arc=1pt,boxsep=2pt,
  before upper={\vphantom{dlg}},colframe=linkcolor,coltext=linkcolor,
  fontupper=\sffamily\bfseries\tiny,colback=white,overlay={\begin{tcbclipinterior}
  \fill[linkcolor] (frame.south west)
  rectangle node[text=white,font=\sffamily\bfseries\tiny,rotate=0]{CPU} 
  ([xshift=6mm]frame.north west);\end{tcbclipinterior}}}
\robustify{\figtimebox}
\pdfstringdefDisableCommands{%
  \def\figtimebox#1{'#1'}%
}
\newcommand{\figtime}[1]{\IfFileExists{figures/#1.py.time}%
{%
\cilink{\figtimebox{\input{figures/#1.py.time}\unskip s}}
}{}}

% Define the `oscaption` command for open source figure captions
\newcommand{\oscaption}[2]{\caption{#2 \codelink{#1} \figtime{#1}}}

% Code examples
\definecolor{codegreen}{rgb}{0,0.6,0}
\definecolor{codegray}{rgb}{0.5,0.5,0.5}
\definecolor{codepurple}{rgb}{0.58,0,0.82}
\definecolor{backcolour}{rgb}{0.95,0.95,0.95}
\lstdefinestyle{mystyle}{
    backgroundcolor=\color{backcolour},
    commentstyle=\color{codegreen},
    keywordstyle=\color{magenta},
    numberstyle=\tiny\color{codegray},
    stringstyle=\color{codepurple},
    basicstyle=\small\ttfamily,
    breakatwhitespace=false,
    breaklines=true,
    captionpos=b,
    keepspaces=true,
    numbers=left,
    numbersep=5pt,
    showspaces=false,
    showstringspaces=false,
    showtabs=false,
    tabsize=2,
    aboveskip=1em,
    belowskip=1em,
    keywords=[2]{map},
    keywordstyle=[2]{\color{black!80!black}},
    upquote=true
}
\lstset{style=mystyle}

% Typography obsessions
\setlength{\parindent}{3.0ex}
\renewcommand\quad{\hskip\fontdimen3\font}

% https://tex.stackexchange.com/a/184474
\usepackage{stackengine,scalerel}
\def\lnlam{\ThisStyle{\ensurestackMath{\stackon[-2.4\LMpt]{%
  \SavedStyle\lambda}{\kern-.5pt\kern\LMpt\rule{1\LMex}{.25pt+.15\LMpt}}}}}

% Bibliography stuff
\bibliographystyle{aasjournal}

% Begin!
\begin{document}

% Title
\title{Title here}

% Author list
\author[0000-0002-0296-3826]{Rodrigo Luger}
\email{rluger@flatironinstitute.org}
\affil{Center~for~Computational~Astrophysics, Flatiron~Institute, New~York, NY}
%

\begin{abstract} 
Abstract here.
%
\href{https://github.com/rodluger/starrynight}{\color{linkcolor}\faGithub}
\end{abstract}

%
\section{Introduction}
%
Intro here \citep{Luger2019}.
Here's an equation with a proof:
%
\begin{proof}{dummy}
    \label{eq:dummy}
    1 + 1 = 2.
\end{proof}
The proof is a Jupyter notebook with a formal derivation of the solution,
an informal justification, a numerical validation, or whatever you want it
to be.
%
And here's a figure with a link:
%
\begin{figure}[h!]
    \begin{centering}
    \includegraphics[width=0.5\linewidth]{figures/dummy.pdf}
    \oscaption{pretty_function}{%
        This is a plot of a pretty function. And at the end of this
        caption is a symbol with a link to the \emph{exact} script
        that generated it, hosted on \textsf{GitHub}.
        \label{fig:dummy}
    }
    \end{centering}
\end{figure}

\section{Integration}

Define the pairwise difference operator
%
\begin{align}
    \label{eq:pairdiff}
    \Delta \bvec{v} \equiv \sum_{i=0}^{2N} (-1)^{i + 1} v_i
\end{align}
%
which sums the difference of successive pairs of values in 
an array $\bvec{v} = \{ v_0, v_1, v_2, v_3, {\cdot\cdot\cdot}, v_{2N}\}$.

\newcommand{\kap}{\boldsymbol{\kappa}}
\newcommand{\kmt}{k^{-2}}

\subsection{The integral $\mathcal{J}$}
%
\begin{align}
    \label{eq:J}
    \mathcal{J}_v(k, \kap) = 
        \int_{\frac{\kappa_i}{2}}^{\frac{\kappa_{i+1}}{2}}
        \sin^{2v}\varphi
        \left(1 - \kmt\sin^2\varphi\right)^\frac{3}{2} 
        \mathrm{d}\varphi
\end{align}
%
Given the lower boundary condition
%
\begin{proof}{J0}
    \label{eq:J0}
    \mathcal{J}_0(k, \kap) &= 
        \frac{1}{3} \bigg(
            2 (2 - \kmt) \Delta \mathbf{E}(k, \kap) + 
            (\kmt - 1) \Delta \mathbf{F}(k, \kap) + 
            \kmt \Delta \mathbf{z}(k, \kap)
        \bigg)
%
    \nonumber \\
%
    \mathbf{z}(k, \boldsymbol{\kappa}) &= 
        \sin\left(\frac{\kap}{2}\right) 
        \cos\left(\frac{\kap}{2}\right) 
        \sqrt{1 - \kmt \sin^2\left(\frac{\kap}{2}\right)}
\end{proof}
%
and an upper boundary condition $\mathcal{J}_N$ 
(computed by numerical integration of Equation~\ref{eq:J}),
the values of all intermediate $J_v$ can be obtained by solving the following
tridiagonal problem:
%
\begin{proof}{Jtri}
    \label{eq:Jtri}
    \begin{pmatrix}
        a_0   & 1     &       &       &       &       \\
        b_1   & a_1   & 1     &       &       &       \\
            & b_2   & a_2   & 1     &       &       \\
            &       & b_0   & a_3   & 1     &       \\
            &       &       & \ddots& \ddots&\ddots \\
            &       &       &       & b_N   & a_N   
    \end{pmatrix}
    \begin{pmatrix}
        \mathcal{J}_1 \\
        \mathcal{J}_2 \\
        \mathcal{J}_3 \\
        \mathcal{J}_4 \\
        \cdot\cdot\cdot \\
        \mathcal{J}_{N-1}
    \end{pmatrix}
    =
    \begin{pmatrix}
        c_0 - b_0 \mathcal{J}_0 \\
        c_1 \\
        c_2 \\
        c_3 \\
        \cdot\cdot\cdot \\
        c_N - \mathcal{J}_N
    \end{pmatrix}
\end{proof}
%
where the recursion coefficients are given by
%
\begin{proof}{Jtri}
    \label{eq:Jtri_coeffs}
    a_v(k) &= -2\frac{(v + 1) + (v - 1) k^2}{2v + 3} \nonumber \\
    b_v(k) &= \frac{(2v - 3) k^2}{2v + 3} \nonumber \\
    c_v(k, \kap) &= \frac{\Delta \mathbf{q_v}(k, \kap)}{2v + 3} \nonumber \\
    \mathbf{q_v}(k, \kap) &= 
        k^2 
        \sin^{2v + 1}\left(\frac{\kap}{2}\right) 
        \cos\left(\frac{\kap}{2}\right) 
        \bigg(1 - \kmt \sin^2\left(\frac{\kap}{2}\right)\bigg)^\frac{5}{2}
\end{proof}

% Bibliography
\bibliography{bib}


\end{document}
